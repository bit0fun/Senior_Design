\documentclass[letterpaper, 11pt, twoside]{article}
\usepackage{longtable}
\usepackage{graphicx}
\usepackage[left=1in, right=1in, top=1.00in, bottom=1.00in]{geometry}
\usepackage{subfigure}
\usepackage{hyperref}
\usepackage{amssymb}

%Header information
\usepackage{fancyhdr}
\pagestyle{fancy}
\fancyhead{}
\fancyhead[CO, CE]{Section \thesection}
\fancyhead[RO, LE]{Low Power Server: Simulator Report}
\fancyfoot{}
\fancyfoot[LO, RE]{\thepage}
\fancyfoot[CE, CO]{Sakhile Mathunjwa}

\hypersetup{colorlinks=false,linktoc=all}
\begin{document}

\begin{titlepage}
	\begin{center}

	%Project Information
	\vspace*{1cm}
	\Huge
	\textbf{Low Power Server}

	\vspace{0.5cm}
	\LARGE
	University of Rochester ECE Senior Design 

	%Title
	\vspace{1cm}
	\LARGE
	Simulator Report
	\vspace{1.5cm}
	
	%Author
	\textbf{Sakhile Mathunjwa} 
	\vfill
	
	
	\end{center}

\end{titlepage}


\cleardoublepage



\tableofcontents
\setcounter{page}{1}
\cleardoublepage

% The overview section should be brief, like an abstract. Try to be concise when 
% talking about what you did. The other parts of the report are for going into more detail
\section[Overview]{Overview of the Simulator}

% This paragraph is the real 'abstract' like part. Just talk about why we made it, why it's useful,
% and pretty much whatever else you want to talk about in a short way. Just don't go that far into detail;
% think of this part as what someone who stumbles across this paper would think as they read this. If they
% are interested in it, they should read further. Throughout the paper point out the sections to look at
% for more detail on a specific subject. I have enabled links, so you can do a link though I don't know the syntax
% off the top of my head.
\paragraph{}

% Here you should expand a bit on actually using it, in a very breif way. Like running the command. Tell the reader
% to go to the appropriate section to learn more about compiling the simulator from the source code.
\subsection{Use Case}
\paragraph{}

% This part is a brief 'what did you do?'. 
\subsection{Design Choices and Functionality}
\paragraph{}


\cleardoublepage

\section[Simulator Usage]{How to use the Simulator}
\paragraph{}

% Not a tutorial, but this should have a step by step guide to compiling the code from source.
% Without this information, the project is effectively useless as no one would be able to use it.
\subsection[Compilation]{How to Compilation}
\paragraph{}

% Now for the fun part, here you should go over how to actually run the program. I would include a bit on 
% using the assembler and linker, to get a binary but you don't have to go into detail as that should be more common
% knowledge. Then all the options for running the program, and what they do (such as when you type 'help' after a command)
% and of course how to actually use the program.
\subsection[Simulator Command]{Running the Simulator}
\paragraph{}


% Here is basically all the bugs that you have found, but have fixes for that can be easily done.
% This should be more about compilation things and actual user error, rather than bugs in the code.
% For example, the 'qmake' issue, would have a soultion of 'make sure the Qt libraries are installed.
\subsection{Troubleshooting}
\paragraph{}

% I would put various bugs you have found ( and commit numbers where you have
% fixed them if they are no longer present ). It's good to acknolwedge what is wrong and ideas on how to fix the issues,
% as it shows the effort put into the work.
\subsection{Known Bugs}
\paragraph{}

\cleardoublepage

% This section is where you would talk about how you made the program, not what was done. Such as the libraries you used,
% how you solved certain problems, why you did what you did. You should be able to defend why you did what you did, if there
% were multiple paths you could have taken. Don't explicitly write the questions, you should infer that a question was asked and
% you are answering it in writing. For example: (assume that a struct was used, instead of an integer for the instruction stream)
% "In order to increase readability for parsing the instructions, a struct was used to separate the bit fields. An added benefit is that this
% eliminates the need to use bitwise operations later in the program". The implied question is "why use a struct instead of an integer?". 
\section[Internals]{Simulator Internals Documentation}

% Here discuss the different parts of the program, and how you put them together. The different parts should be explained later in
% their respective sections
\paragraph{}


\subsection{ELF Handling}
\paragraph{} 

\subsection{CPU Simulation}
\paragraph{}

\subsection{Memory Simulation}
\paragraph{}

%Even without a Graphical User Interface, there is still one so you should explain that
\subsection{User Interface}
\paragraph{}

%add any additional sections I may have missed. Don't feel weird about changing the sections around, or splitting them into further
%sections. There should be subdivisions of what you did, but don't make them so divided that it's impossible to read


\end{document}
